\chapter{Introduction}

\paragraph{}
Baja SAE is an international collegiate competition run by the Society of Automotive Engineers (SAE) \cite{SAE} where teams design, build, test, and compete with offroad baja-style vehicles.
In the United States, there are three competitions each year across the country for teams to compete in. 
There are three categories of events for which teams are scored: static events, dynamic events, and the endurance event.  
Static events include challenges to test a team's ability to effectively communicate design choices and business aspects of making a vehicle.  
Dynamic events test the abilities of the vehicle to perform in different conditions, including an acceleration event, a maneuverability event, a suspension event, and a traction event.  
Finally, the endurance event tests the vehicle's and driver's ability to withstand rough terrain while racing against other teams' cars for four hours.
At the end of the competition, all the event scores are added together to determine the top-three overall teams at the competition.
To win the competition, it is crucial to perform well in all three categories of events.

\paragraph{}
Cal Poly Racing \cite{CalPolyRacing} competes in the Baja SAE series of competitions.
Over the last several years, Cal Poly Racing has had moderate success, with several top three trophies in dynamic events.
The Baja vehicle runs the series' only dually actuated electronic CVT, an electronically controlled hydraulic clutch for 4WD, and a custom dashboard.
Additionally, the car runs with freewheels in the front, an aero package, and Genesis Racing shocks with a custom anti-roll bar.

\paragraph{}
To design a vehicle capable of withstanding the harsh events in a Baja SAE competition, it is critical to fully understand how every system of the car is behaving.
Understanding the load cases, such as impacts, is essential to making informed design choices and considerations.
To understand these, a Data Acquisition System (DAQ) is needed to log and process sensor data.

\paragraph{}
In this senior project, a DAQ was created on the same platform as the rest of the vehicle that can be used to log data on the car from all control systems as well as sensor aggregation units.
Additionally, the DAQ is able to transmit data wirelessly to monitor the vehicle state remotely.
With a DAQ, design engineers for the team will be able to collect data to influence designs for future years and diagnose issues.
Additionally, electrical engineers on the team will be able to design new sensor units that can be used in the future to collect even more data.
