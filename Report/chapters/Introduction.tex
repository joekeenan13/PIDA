\chapter{Introduction}

\paragraph{}
Baja SAE is an international collegiate competition run by the Society of Automotive Engineers (SAE) where teams design, build, test, and compete with offroad baja style vehicles.
In the United States, there are three competitions held each year across the country for teams to compete in. 
There are three categories of events for which teams are scored: static events, dynamic events, and the endurance event.  
Static events include different challenges to test a team's ability to effectively communicate design choices and business aspects of making a vehcile.  
Dynamic events test the abilities of the vehicle to perform in different conditions.  
These events include an acceleration event, a manueverability event, a suspension event, and a traction event.  
Finally, the endurance event tests the vehicles and drivers ability to withstand rough terrain and wheel to wheel racing for a full four hours.
At the end of the competition, all the event scores are added together to determine who the top three overall teams at the competition are.
To win the competition, it is crucial to perform well in all three styles of events.

\paragraph{}
Cal Poly Racing competes in the Baja SAE series of competitions.
Over the last several years, Cal Poly Racing has had a moderate amount of success, with several place tropies in dynamic events.
The Baja vehcile runs the series only dually actuated electronic CVT, an electronically controlled hydraulic clutch for 4WD, and a custom dashboard.
Additionally, the car runs with freewheels in the front, an aero package, and Genesis Racing shocks with a custom anti-roll bar.

\paragraph{}
In order to design a vehcile capable of withstanding all the harsh events in a Baja SAE competition, it is critical to fully understand how every system of the car is behaving.
Additionally, undertstanding the load cases, such as impacts, is also essential to making informed design choices and considerations.
To understand the the vehicle and the load cases, a data collection system is needed to log and process sensor data.
Data Acquisition Systems (DAQs) are tools that allow for the logging and processing of data.

\paragraph{}

