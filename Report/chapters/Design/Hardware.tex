\section{Hardware Design}

\paragraph{}
The hardware design, with the exception of component selection, was largely outside the scope of this project and was completed by other members of the competition team.
As the most senior member of the electronics team and the Electrical Technical Director, though, part of my responibilities to the project included design reviews and feedback on schematic design, layout, and routing.

\paragraph{}
As with any embedded systems project, the code for the project does depend on the hardware design.
Different periphreals of the microcontroller being used and sensors connected drive different parts of the firmware implementation.
The specific periphreals that are heavily utilized are CAN FD, UART, I2C, SDIO, and ADC.

\subsection{Component Selection}

\paragraph{}
The first major component selected was the STM32H750VBT6 that was discussed in the system design section.
With this microcontroller, there was access to all the desired periphreals for all the necessary communication.
This microcontroller does come in a somewhat large package, being a QFP100 chip.
The pins are all 3.3V tolerant, meaning that 5V analog sensors need to be lowered to have a range that does not exceed 3.3V and all digital IO of periphreals need to operate on 3.3V.
There are more pins available than needed, leaving the potential for adding additional functionailty in future years as new features are desired.

\paragraph{}
To communicate with the CAN bus, a CAN FD transceiver is required.
The CAN controller exists internal to the microcontroller, but it does not include a CAN transceiver internally.
The CAN transceiver needs to support at least a 5 Mbps data rate to support the CAN throughput discussed in the system design section and needs to support the 3.3V IO voltage of the microcontroller.
The TCAN1043ADYYRQ1 chip from Texas Instruments was selected.
This chip can support up to a 8 Mbps data rate and can be configured to use 3.3V for all of the digital IO.

\paragraph{}
Similarly, a USB-UART transceiver was selected to support a USB connection from a user to the DAQ.
Just like the CAN transceiver, the digital IO that physically connects to the microcontroller must support a 3.3V voltage.
The FT231XS-R chip from FTDI was selected due to its popularity and large amount of documentation.
Additioanlly, this chip includes an EEPROM that contains information about the device.
This can be programmed to include identifying information about the device.

\paragraph{}
The GPS selected was the NEO-M9N-00B module from u-blox.
This module was selected because it was utilized in a prior senior project and supports several different communication protocols.
That senior project focused on creating an all-in-one solution for tracking vehicle orientation and heading using an IMU, GPS, and magnetometer.
This also includeds code for interfacing with these sensors.
This senior project was also a useful reference for good practices for routing the RF signal from the GPS to the antenna.

\paragraph{}
A different IMU that was different from the prior senior project was selected.
The main criteria for the IMU was that it had community support for interfacing with it, could operate with IO voltages of 3.3V, and could measure $\pm$3 G's of acceleration.
The peak possible acceleration is approximately 10 G's and occurs when the vehicle stops immediately due to a crash or large, unmovable obstacle.
According to the mechanical team, the vehicle should not experience accelerations of more than 3 G's under typical operation.
The LSM6DSLTR IMU from ST fit all of these criteria.

\paragraph{}
The final major component selected was the micro SD card reader.
A push-in/push-out reader was desired as it helps apply mechanical retention to the SD card.
The vibrational load on the system during operation is non-neglible, so it is important to mechnaically retain the SD card to improve the overal reliability of the system.
The 2201778-1 micro SD card reader from TE Connectivity was selected due to TE Connectivity's high amount of reliability and meeting all the other needs of the component.

\subsection{Schematic Design}

\paragraph{}
As mentioned above, the schematic design was largely out of the scope of this project.
Despite this, there are a few important details regarding the schematic design that are important to cover due to how they impact the firmware design and reliability of the system.

\paragraph{}
The first important aspect of the schematic design is the pinout used for the STM microcontroller.
This pinout can be seen in INSERT FIGURE and was generated and validated by utilizing the STM32CubeIDE IOC tool and tripple checking with the datasheet.
Throughout the testing process, there were a few pinout issues that were discovered that delayed software testing of some key components such as the SD card reader.
These issues were eventually resolved with a second revision of the board.

\paragraph{}
The next major aspect of the schamtic design was to ensure that all directional communication signals were connected to the apprioate pins between the microcontroller and the periphreals.
Specifically, it was important to ensure that for both UART connections, the TX pin of the microcontroller was connected to the RX pin of the apprioate component.
On the CAN controller to CAN transceiver connection, though, the opposite needs to occur.
The CAN TX pin on the microcontroller for the CAN controller must be connected to the CAN TX pin of the CAN transceiver and vice versa for RX.
This can be seen in INSERT FIG and INSERT FIG.

\paragraph{}
It is also important that pull-up and pull-down resistors are connected to the apprioate signals for communication lines and control signals.
Specifically, pull up resistors needed to be added for the I2C communication lines as well as the SDIO data lines.
Additionally, the DAQ acts as a termniating node in the CAN netowrk, so the termination resistor for the CAN bus must be present.

\paragraph{}
The final important aspect for this at a system level was to ensure that the controlled impedances of signals were configured if they were needed.
Speifically, the the CAN spec and USB spec detail controlled differential impedances for their signals.
As a result, these signals needed to be marked as differential pairs on the schematics as seen in INSERT FIG and INSERT FIG.

\subsection{Layout and Routing}