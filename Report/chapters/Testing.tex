\chapter{System Testing and Analysis}
With the system fully designed, the next step is to confirm that all systems are operating correctly.
There were several methods used to confirm all these results, including unit tests, inspection via the debugger, and simple test programs to confirm functionality on hardware.

\section{SD Testing}

\paragraph{}
The first part of this project that needed to be tested was the creating and writing to files on the SD card.
In order to do this, a simple test program creating a csv file with dummy data was made.
During this testing, there were several issues discovered.
The first was that the STM32SD library does not define the function f\_unmount even though it is used in the library.
This function is typically a wrapper around f\_mount and passing in 0's for the filesystem and option.
By updating the f\_unmount function call to a f\_mount with the added paramaters, the library worked fine.

\paragraph{}
After the issue with the library was resolved, the next issues were related to the hardware.
The pinout for the SDIO interface had some conflicts resulting in the SD card being connected to non SDIO pins.
This issue required a respin of the hardware to fix, delaying testing by several weeks.
During this time, attempts were made to rework the PCB to connect the SD card to the correct microcontroller pins.
It was discovered while attempting to run the test program that there were stability problems with the reworked connections.
There were a few instances where the SD object would initialize properly, inidicating that the library was working successfully.
Once the respinned board arrived, it was confirmed that the SD library was functioning and the test program sucessfully created a dummy csv file.

\section{Checksum Testing}

\paragraph{}
In order to confirm that the checksum calculation used in logging and radio communication was functioning as expected, a unit test was created.
There were 5 different values tested, and the checksum was computed by hand and then compared to what was calculated by the DAQ.
As can be seen in \cref{tab:ChecksumTesting}, the checksum was computed correctly in all tests.

\begin{table}[H] \label{tab:ChecksumTesting}
\caption{Checksum Unit Testing Results}
\centering
\begin{tabular}{c c c c}
\hline\hline
Test & Value & Hand Computed & DAQ Computed \\ [0.5ex]
\hline
1 & 0x0 & 0 & 0 \\
2 & 0xAFAFAFAFAF & 0 & 0 \\
3 & 0x123456789ABCDEF & 0 & 0 \\
4 & 0xF000 & 0 & 0 \\
5 & 0xFFFFFFFF & 0 & 0 \\ [1ex]
\hline
\end{tabular}
\end{table}

\section{Metadata Testing}

\paragraph{}
The metadata generation for files is a critical feature for being able to properly decode the data into something useful for design engineers to use.
To confirm that the metadata is being generated properly, the file metadata must first be computed by hand.
Then, since the SD card functionality has been confirmed to work previously, the file metadata can be written to a file on the SD card.
This file can be viewed in a hex editor to see each byte of the file, where it was confirmed to match the metadata generated by hand.

\section{CAN Testing}

\paragraph{}
There were two parts of CAN that needed to be tested.
The first was that two or more CAN nodes were able to communicate, or that the DAQ was able to communicate with any other node.
To accomplish this, another system designed for the baja car was used.
These two boards were wired together and a CAN test program was written to confirm that both transmitting and receiving functioned fully.
To start, an internal loopback test was performed, followed by an external loopback test, and finally completing a normal test between two nodes.
This test was completed by using a fixed ID of 1, a data length of 32 bytes, and the data section was filled with incrementing values starting from 0 and ending at 31, with each value corresponding to the index of the byte array of the message.

\paragraph{}
After the CAN bus was confirmed to work and data was transmitted, CAN loading also needed to be tested.
This could be done empirically, knowing the total amount of data being transmitted in each message and the frequency that each message was being sent at.
To confirm that the desired throughput was actually be achieved, the total number of each CAN ID was tracked over 10 seconds to confirm that the correct number of CAN messages were being received.

\begin{table}[H] \label{tab:CANTesting}
\caption{CAN Testing Results}
\centering
\begin{tabular}{c c c c c c}
\hline\hline
Message & Frequency & Data & 1 second & 5 seconds & 10 seconds \\ [0.5ex]
\hline
1 & 400 hz & 36 Bytes & 400 & 2000 & 4001 \\
2 & 400 hz & 48 Bytes & 400 & 2000 & 4001 \\
3 & 400 hz & 48 Bytes & 400 & 2000 & 4001 \\
4 & 100 hz & 8 Bytes & 100 & 500 & 1000 \\
5 & 10 hz & 16 Bytes & 10 & 50 & 100 \\
6 & 10 hz & 8 Bytes & 10 & 50 & 100 \\
7 & 100 hz & 20 Bytes & 100 & 500 & 1000 \\ [1ex]
\hline
\end{tabular}
\end{table}

\paragraph{}
As can be seen in \cref{tab:CANTesting}, the CAN bus was not overloaded.
This is the expected result and supports the previously computed CAN loading and confirms that there are no issues with the amount of data being transmitted.
Additionally, this implies that there is room for additional nodes to send data, leaving room for different sensor expansions for permanent or temporary sensing units.

\section{Radio Testing}

\paragraph{}
The last system to test was the radio communication.
In order to test that this task is working as expected, the digi XTCU tool can be used to read all the data received over the air.
To confirm correct operation, a similar test used in the CAN testing was performed, where several bytes were written to the transmitting radio with incrementing values.
On the receiving side, these bytes are able to be viewed in the XTCU tool, confirming that all bytes being sent were received.

\paragraph{}
After confirming that all bytes being transmitted were being sent and received properly, the packet construction needed to be confirmed.
Since the checksum calculation was confirmed previously, the packet creation can be confirmed in the debugger.
On one iteration, a packet was constructed by hand and then confirmed by utlizing the debugger and looking at the bytes of the created packet.
It was confirmed that packets were being constructed correctly.

\paragraph{}
Finally, the radio's line of sight distance was tested.
This was done by driving the DAQ further and further away from the base antenna until no packets were being received at the base side.
The base side remained at the aerospace hangar on campus (Bldg 004) and the car was driven further down Highland Dr and performance was measured at varying distances.
As can be seen from \cref{tab:RadioDistance}, as the distance increased, the amount of dropped packets also increased.  Once the car got 2.5 miles away, the amount of packets being dropped became unacceptable.
This testing was done with as close to line of sight as was possible, but there were not any fully clear lines of sight going as far as needed, so there are some interferance due to trees and elevation change.

\begin{table}[H] \label{tab:RadioDistance}
\caption{Radio Distance Testing Results}
\centering
\begin{tabular}{c c c}
\hline\hline
Distance (mi) & Packets Dropped (\%) & Acceptable (Y/N) \\ [0.5ex]
\hline
1 mile & 12\% & Y \\
1.5 miles & 20 \% & Y \\
2 miles & 33 \% & Y \\
2.5 miles & 85 \% & N \\
3 miles & 100 \% & N \\ [1ex]
\hline
\end{tabular}
\end{table}
