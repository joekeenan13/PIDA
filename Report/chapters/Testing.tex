\chapter{System Testing and Analysis}

\subsection{SD Testing}

\subsection{Data Testing}

\subsection{CAN Testing}

\paragraph{}
There were two parts of CAN that needed to be tested.
The first was that two or more CAN nodes were able to communicate, or that the DAQ was able to communicate with any other node.
To accomplish this, another system designed for the baja car was used.
These two boards were wired together and a CAN test program was written to confirm that both transmitting and receiving functioned fully.
To start, an internal loopback test was performed, followed by an external loopback test, and finally completing a normal test between two nodes.
This test was completed by using a fixed ID of 1, a data length of 32 bytes, and the data section was filled with incrementing values starting from 0 and ending at 31, with each value corresponding to the index of the byte array of the message.

\paragraph{}
After the CAN bus was confirmed to work and data was transmitted, CAN loading also needed to be tested.
This could be done empirically, knowing the total amount of data being transmitted in each message and the frequency that each message was being sent at.
To confirm this calculation, 
