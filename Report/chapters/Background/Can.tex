\section{CAN}

\paragraph{}
Controller Area Network (CAN) busses are commonly used in automotive applications to connect different control or instrumentation nodes together.
This allows for any node to communicate with any other node on the bus.
CAN utilizes a two wire asynchronous differential twisted pair signal to transmit across the bus.
The asynchronous nature of this protocol reduces the number of wires required to transmit data.
By utilizing a differential twisted pair, noise and interference are reduced improving reliability and robustness of the network.
However, only utilizing a single differential pair means that a node can only transmit or only receive at any given time, reducing throughput.

\paragraph{}
CAN is an addressed based communication protocol.
An address can correspond to a specific node or to a specific message.
Since CAN only utilizes a signle differential signal, it must negotiate to determine which node is transmitting and which nodes are receiving.
The lowest address trying to be transmitted wins the negotiation, meaning that priority can be assigned to messages by assigning a lower value for an address to the message.

\subsection{CAN 2.0}

\subsection{CAN FD}